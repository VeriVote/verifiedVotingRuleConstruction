\documentclass[11pt,a4paper,notitlepage]{report}
\usepackage[utf8]{inputenc}
\usepackage[T1]{fontenc}
\usepackage{lmodern}
\usepackage{sectsty}
\usepackage{xpatch}
\usepackage{csquotes}
\usepackage{upquote}

\makeatletter
\xpatchcmd{\@maketitle}
  {\@title}
{%\vspace{-2.45\baselineskip}
\bfseries\@title\vspace{.25\baselineskip}}
  {}{}
\makeatother

\makeatletter
\providecommand{\institute}[1]{%
    \apptocmd{\@author}{\end{tabular}%
    \vspace{.5\baselineskip}\par
    \begin{tabular}[t]{c}
    #1}{}{}
}
\makeatother

\newcommand{\email}[1]{%
    \texttt{\href{mailto:#1}{\color{black}{#1}}}%
}

\pretocmd{\tableofcontents}{\sectionfont{\Huge}}{}{}
%\let\clearpage\relax
\apptocmd{\tableofcontents}{\sectionfont{\Large}}{}{}

\usepackage{isabelle,isabellesym}
\usepackage{amssymb} % amsfonts
\usepackage{pdfsetup}

\urlstyle{rm}
\isabellestyle{it}

% for uniform font size
%\renewcommand{\isastyle}{\isastyleminor}

% This is required due to some unusual behaviour by Isabelle.
% It should be included in isabellesym.sty automatically, but it is not.
\newcommand{\isasymcirclearrowleft}{\isamath{\circlearrowleft}}

\begin{document}

\title{Verified Construction of Fair Voting Rules}
\author{Michael Kirsten}
\institute{Karlsruhe Institute of Technology (KIT), Karlsruhe, Germany\\
\email{kirsten@kit.edu}}
\maketitle

\begin{abstract}
Voting rules aggregate multiple individual preferences in order to make a
collective decision.
Commonly, these mechanisms are expected to respect a multitude of different
notions of fairness and reliability, which must be carefully balanced to avoid
inconsistencies.

This article contains a formalisation of a framework for the construction of
such fair voting rules using composable modules~\cite{adt2019,lopstr2019}.
The framework is a formal and systematic approach for the flexible and verified
construction of voting rules from individual composable modules to respect such
social-choice properties by construction.
Formal composition rules guarantee resulting social-choice properties from
properties of the individual components which are of generic nature to be
reused for various voting rules.
We provide proofs for a selected set of structures and composition rules.
The approach can be readily extended in order to support more voting rules,
e.g., from the literature by extending the sets of modules and composition
rules.
\end{abstract}

\tableofcontents

\parindent 0pt\parskip 0.5ex

\input{session}

% optional bibliography
\bibliographystyle{plainurl}
\bibliography{root}

\end{document}
